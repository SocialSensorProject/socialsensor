%!TEX root = document.tex

\label{sec:introduction}

Twitter hosts lots of information, on average more than $2,200$ new tweets every second. This can get up to $3$ to $4$ times increase during large events such as tsunami. \footnote{\hyperref[]{https://blog.twitter.com/2011/the-engineering-behind-twitter-s-new-search-experience}}
\begin{itemize}
\item Twitter is a vast sensor of content generated by latent phenonema (e.g., flu, political sentiment, elections, environment).
\item Learning topical social sensors (politicians in NY, road conditions in Toronto) -- very broad topics for which its hard to manually specify a useful query.
\item But there is interesting topical content and wouldn't it be cool if we could learn a social sensor for a targeted topic?
\item Key insight is that hashtags are topical and can be used to bootstrap a supervised learning system that as we will show generalizes well beyond the seed hashtags.
\item Conclusion is a new way to build topical real-time feeds that are otherwise difficult to do with existing Twitter tools (???).
\end{itemize}
section{Learning Topical Social Sensors}

Start off with the questions that we want to answer in this section:

- How to evaluate, labeling (problem of no supervised labels for tweets, indirect via hashtags as topical surrogates, leads to question of hashtag curation)?

- Which classification algorithm is best / most robust for learning topical social sensors?
