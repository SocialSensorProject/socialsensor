This work fills a major gap in event detection and tracking from
social media on identifying emerging topics from long-running themes
with minimal user supervision.  We contribute a novel supervised
method for training social sensors with minimal user curation by using
a small seed set of hashtags as topical proxies for automatic
supervised data labeling.
%The supervised classification and ranking
%methods learn topical content from a large feature space. We train our
%social sensor on known topical content, but tune it on novel topical
%validation content which ensures optimal generalization. The
%experiments are on a corpus of over $800$ million tweets crawled with
%Twitter Search API.
Our results suggest that these learned social sensors generalize well
to unseen future topical content and provide a novel paradigm for the
extraction of high-value content from social media. Furthermore, an
extensive analysis of features and feature attributes across different
topics has revealed two key insights: (1)~largely independent of
topic, simple terms are the most informative feature followed by
location features and that (2)~the number of unique hashtags and
tweets by a user correlates more with their informativeness than their
follower or friend count.

Among many interesting directions, future work should explore the
following enhanced topical social sensor learning tasks: (1)
optimizing rankings not only for topicality but also to minimize the
lag-time of novel content identification, (2) optimizing queries for
boolean retrieval oriented APIs such as Twitter, and (3) utilizing
more social network structure to exploit a more expressive graph-based
features.  Altogether, we believe this and future work will pave
the way for a new class of social sensors that learn to
identify broad themes of topical information with minimal user
interaction and enhance the overall social media user experience.
