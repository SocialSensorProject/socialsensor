% This section covers the *formal* framework for learning topical social sensors

% What are we doing?  Formal problem setup.
% - Tweets, features, labels, classification problem definition (for a generic classifier).
% - How we label data.
% - How we train (validation set critical for hyperparameters, what metric used for selection?).
% - Note that formal performance evaluation provided in experimental section.

Perhaps the critical bottleneck for learning targeted topical social
sensors is to achieve sufficient supervised content labels.  With
data requirements often in the thousands of labels to ensure effective
learning and generalization over a large candidate feature space as
found in social media,
manual labeling is simply too time-consuming for many users and
crowdsourced labels are both costly and prone to misinterpretation of
users' information needs.  Fortuitously, hashtags have emerged in
recent years as a pervasive topical proxy on social media sites ---
hashtags originated on IRC chat, were adopted later (and perhaps most
famously) on Twitter, and now appear on other microblogs (e.g., Sina
and Tencent Weibo) and even Facebook.  Hence as a simple enabling
insight that serves as a catalyst for effective topical social sensor
learning, we leverage a (small) set of user-curated topical hashtags
to efficiently provide a large number of supervised topic labels for
social media content.

With the data labeling bottleneck resolved, we proceed to train
supervised classification and ranking methods to learn topical content
from a large feature space of source users and their locations, terms,
hashtags, and mentions.

