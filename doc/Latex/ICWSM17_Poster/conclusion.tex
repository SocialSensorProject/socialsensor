%!TEX root = icwsm2017.tex

This work provides a long-term study of topic classifiers on Twitter that further justifies classification-based topical filtering approaches while providing detailed insight into the feature properties most critical for topic classifier performance.
%
%This work fills a major gap in event detection and tracking from
%social media on identifying emerging topics from long-running themes
%with minimal user supervision.  We contribute a novel supervised
%method for training social sensors with minimal user curation by using
%a small seed set of hashtags as topical proxies for automatic
%supervised data labeling.
%The supervised classification and ranking
%methods learn topical content from a large feature space. We train our
%social sensor on known topical content, but tune it on novel topical
%validation content which ensures optimal generalization. The
%experiments are on a corpus of over $800$ million tweets crawled with
%Twitter Search API.
Our results suggest that these learned topical classifiers generalize well
to unseen future topical content over a long time horizon (i.e., one year)
and provide a novel paradigm for the
extraction of high-value content from social media. Furthermore, an
extensive analysis of features and feature attributes across different
topics has revealed key insights including the following two: 
(i) largely independent of
topic, generic terms are the most informative features followed by
topic-specific locations, and (ii) the number of unique hashtags and
tweets by a user correlates more with their informativeness than their
follower or friend count.

Among many interesting directions, future work might evaluate a range of 
topical classifier extensions: (1)
optimizing rankings not only for topicality but also to minimize the
lag-time of novel content identification, (2) optimizing queries for
boolean retrieval oriented APIs such as Twitter, (3) identification of 
long-term temporally stable predictive features, and (4) utilizing
more social network structure as graph-based 
features.  Altogether, we believe these insights will facilitate 
the continued development of effective topical classifiers for Twitter that learn to
identify broad themes of topical information with minimal user
interaction and enhance the overall social media user experience.
