% Contribution 
% 1. What do we offer? Combing recent ideas, expand, offer more
% 2. What was the previous ideas? 
%      2.1. Automatic labeling of lots of tweets, 
%      2.2. Using learning methodologies for detection of general topical content
% 3. How did we build on these ideas?
% 4: What do we offer in addition to combination.
\subsection*{Discussion of Contributions}
In this work, we coalesce recent ideas on learning social sensors for general topic detection. We expand these works to learn a generalizable supervised method with minimal user curation for detecting and ranking topical content over a variety of topics and on a long-term dataset. We believe that no earlier work covers all the aspects of the work presented here.
Earlier works discuss leveraging learning methodologies for detection of topical content from social media for general topics ~\cite{lin2011smoothing,yang2014large,magdy}. One of the key challenges of using learning methods for general topics and large number of tweets is automatic labeled data aquisition. To this purpose~\cite{lin2011smoothing} discuss automatic labeling of tweets by using one hashtag as topic proxy.~\cite{magdy} use a user-defined query to label tweets and~\cite{yang2014large} take a co-training approach based on embedded URLs in the tweet and tweet text to label tweets. We build and extend on~\cite{lin2011smoothing}'s idea of automatic labeling of tweets, however we choose a \emph{set} of hashtags for each topic instead of a single hashtag which we will show to be imperative for evaluating generalization. To learn social sensors for general topic detection,~\cite{lin2011smoothing} use information retrieval method (language models),~\cite{yang2014large} take advantage of topic modeling techniques and~\cite{magdy} apply SVM classifier. Here, we leverage more than one supervised learning method for the purpose of detection and ranking of topical content. We present a unique method for splitting hashtags and Twitter data that encourages generalization to new unseen future content. 

%%%%% TRACKING GENERAL TOPICS
\subsection*{Tracking General Topics} 
represents use of social media sensors for detecting and tracking general topics such as "Baseball" and "Fashion". Researchers have collected labeled data by using a single hashtag for each topic~\cite{lin2011smoothing}, a user-defined query for each topic~\cite{magdy}, or co-training based on the URLs and text of the tweet~\cite{yang2014large}.~\cite{lin2011smoothing} leverages language models to train models using unigrams and bigrams,~\cite{magdy} applies SVM classifier on extracted hashtags, unigrams, users and mentions as features, and~\cite{yang2014large} defines the problem as topic modeling of tweets. We expand on~\cite{lin2011smoothing}'s work and use a set of hashtags instead of a single hashtag. We extract hashtags, mentions, unigrams, users as features inline with these works. However, we add location as another feature which we will show later that location is the second most important feature for detection of topical content. We take advantage of various supervised learning methods and provide a novel framework for learning in terms of splitting the data and hashtags as topical proxies that would ensure generalization to future unseen content. While these works provide a good basis for this work, there are many fine-grain but important differences between previous works and this work with the most important ones being:
\begin{enumerate}
\item We analyzed long-term sensor performance on detecting topical content over two years of Twitter data and across a variety of topics.
\item We provide a novel and clear framework for splitting hashtags to train, validation and test in a way ensuring generalization to future unseen content.
\item We present ranking in addition to correct classification while none of the other works provide ranking.
\item We deliver a comprehensive longitudinal study on features and their attributes over two years of tweets that supports our insights for learning and relevance of features to topicality while these works had little or none analysis over their features.
\item We extract \textit{Location} as one of the features which none of these works do and as we show in our feature analysis, \textit{Location} is the second most important feature beating even hashtags in terms of correlation with topicality.
\end{enumerate}
