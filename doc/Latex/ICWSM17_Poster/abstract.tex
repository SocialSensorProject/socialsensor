%!TEX root = icwsm2017_poster.tex

Twitter represents a massively distributed information source over a kaleidoscope of topics ranging from social and political events to entertainment and sports news.  While recent work has suggested that variations on standard classifiers can be effectively trained as topical filters~\cite{lin2011smoothing,yang2014large,magdy}, there remain many open questions about the efficacy of such classification-based filtering approaches.  For example, over a year or more after training, how well do such classifiers generalize to future novel topical content, and are such results stable across a range of topics?  Furthermore, what features and feature classes are most critical for long-term classifier performance?  To answer these questions, we collected a corpus of over 800 million English Tweets via the Twitter streaming API during 2013 and 2014 and learned topic classifiers for 10 diverse themes ranging from social issues to celebrity deaths to the ``Iran nuclear deal''.  The results of this long-term study of topic classifier performance provide a number of important insights, among them that (1)~such classifiers can indeed generalize to novel topical content with high precision over a year or more after training and (2)~simple terms and locations are the most informative feature classes (despite training on classes labeled via hashtags).  
%In summary, this work provides a long-term study of topic classifiers on Twitter that further justifies classification-based topical filtering approaches while providing insight into the feature properties most critical for topic classifier performance.