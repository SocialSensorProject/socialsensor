% Scott's outline
%
% EU Social Sensor project: what to say?  Many works integrated
%   into following text, but none specifically learning topical sensors
%   from supervised labels.
%
% Trending topics: learning but not supervised
%
% RT recommendation: RTs are personalized but not topical
%
% Specific events: predicting earthquakes.
%
% Friend sensor: friendship paradox, focusing of friends-only, no supervised
%                training for a topic, just trying to pick informative users
%                however: our work suggests users are least important features
%                (Figure 2)

The concept of social media as a sensor is prevalent in the literature
and in this section we survey four related areas of active research:
(1) trending topic detection, (2) tweet recommendation, (3) friend
sensors, (4) and specific event detection such as earthquake or
influenza sensors.  Despite the partial overlap and superficial
similarities between this paper and related social sensor work, we
argue that no prior work has learned targeted social sensors for
arbitrary topics using supervised learning methods as done in this
paper.
%In the following, we explain why each of these four areas is different from
%the social sensor learning work proposed here.

\vspace{2mm}
\noindent {\bf Trending Topic Detection} represents one of the most popular
types of social sensor and can be subdivided into many categories.  The
first general category of methods define trends as topically coherent
content and focus on clustering across lexical, linguistic, temporal
and/or spatial
dimensions~\cite{petrovic,ishikawa,murata,becker,tweetmotif,wangLee}.
%\cite{wei} proposed a
%graphical model to discover latent events clustered in the spatial,
%temporal and lexical dimensions, while \cite{yamamoto} focused on the
%task of multi-label classification of tweets into living aspects such
%as eating.
The second general category of methods define trends as temporally
coherent patterns of terms or keywords and focus largely on detecting
bursts of terms or
phrases~\cite{mathioudakis,cuiZhang,zhaoSports,nichols,aiello}.
% I don't know that this means... aren't all of the above methods user-defined
% criteria as well? -SPS
%
%The third category, query-based methods, focuses on the hypothesis that trending
%topics can be detected by measuring user defined
%criteria \cite{albakour,sakakiDrive}.
The third category of methods extends the previous categories by
additionally exploiting network structure properties~\cite{budak}.
Despite this important and very active area of work that can be
considered a type of social sensor, trending topic detection is
intrinsically unsupervised and not intended to detect targeted topics.
In contrast, the work in this paper is based on supervised learning of
a specific topical social sensor derived from the topical set of
hashtags provided by the user.

% From EU Social Sensor Project...
%
% NOTE: Whoa... this is everything yet nothing... not worth discussing. -SPS
%The final category, hybrid method
%of \cite{diplaris} introduced concept of Dynamic Social Containers in
%this work to take advantage of aggregation of mining both the
%structure, content, and multimedia data to index and provide
%personalized, context-aware search. In this work, the authors defined
%social sensor as analyzing the dynamic and massive amount of
%information provided by user with the purpose of extracting unbiased
%trending topics and events in addition to using social connections for
%recommendation.

% From EU Social Sensor Project...
%
% This is just a TF-IDF term-based method, not worth a separate discussion
%With the purpose of comparison of methods, \cite{aiello} evaluated six
%trending topic detection methods on three Twitter datasets differing
%in time scale and topic churn rate. The authors conclude that natural
%language processing techniques perform well on focused
%topics. However, techniques mining temporal distribution of concepts
%are needed to handle more heterogeneous streams.

% Zahra: trending topics are unsupervised... we're supervised. -SPS
% 
%However, trending topics detection methods are not targeted. Our
%method differs from trending topic detection methods in that we are
%focusing on a set of topics that cannot necessarily be detected using
%bursts.Thus, trending topics detection methods are of limited
%relevance to the work presented hereinafter.

\vspace{2mm}
\noindent {\bf Tweet Recommendation} represents an alternate use of
social sensors and falls into two broad categories: personalized or
content-oriented recommendation and retweet recommendation.  For the
first category, the objective of personalized recommendation is to
observe a user's interests and behavior from their user profile,
sharing or retweet preferences, and social relations to generate
tweets the user may like~\cite{Yan,chen}.  The objective of
content-oriented recommendation is to use source content (e.g., a news
article) to identify and recommend relevant tweets (e.g., to allow
someone to track discussion of a news article)~\cite{Krestel}.  For
the second category, there has been a variety of work on retweet
prediction that leverages retweet history in combination with
tweet-based, author-based, and social network features to predict
whether a user will retweet a given
tweet~\cite{can,xu,petrovicOsborne}.  Despite the fact that all of
these methods recommend tweets, they --- and recommendation methods in
general --- are not focused on a specific topic but rather on
predicting tweets that correlate with the preferences of a specific
user or that are directly related to specific content.  Rather the
focus with learning topical social sensors is to learn to predict for
a broad theme (independent of a user's profile) in a way that
generalizes beyond existing labeled topical content to novel future
topical content.

%Another set of studies have moved towards creating more generalizable
%methods. Using a dataset of 55,000 news articles and 121,000
%tweets, \cite{Krestel} compared four different methods of language
%model, topic model, logistic regression, and boosting, to evaluate
%recommended tweets for a given news article.. \cite{Yan,chen} also
%focused on tweet recommendation. Their methods considered the user’s
%twitter profile, including tweet and retweet history, and social
%relations as features. Coupled with tweet popularity, the methods are
%able to generate tweet recommendations. With the purpose of photo
%recommendation on social media websites, \cite{chiarandini} analyzed
%the user logs of pageviews, navigation patterns between
%photostreams. The authors used collaborative filtering method and
%built a stream transition graph to analyze common stream topic
%transitions to this end.

%On retweet prediction, \cite{can,xu,petrovicOsborne} used
%classification-based approaches using tweet-based and author-based
%features. However, \cite{can} took advantage of visual cues from
%images linked in the tweets, and \cite{xu} employed social-based
%features in addition to tweet author-based features. Different from
%the other two works, \cite{xu} performed the analysis from the
%perspective of individual users. \cite{petrovicOsborne} worked on
%retweet prediction of real-time tweeting with online learning
%algorithms and claimed that performance is dominated by social
%features, but that tweet features add a substantial boost. These
%studies showed that temporal features have a stronger effect on
%messages with low and medium volume of retweets compared to highly
%popular messages, and user activity features can further improve the
%performance marginally.

%%%%% TARGETED SPECIFIC TOPIC DETECTION
\vspace{2mm}
\noindent {\bf Specific Event Detection} builds social sensors
as we do in this work but focuses on highly specific events 
such as disasters or epidemics.  For the use case of earthquake
detection, an SVM can be trained to detect earthquake events
and coupled with a Kalman filter for localization~\cite{sakakiEq2}.
%
%% Decriptive studies irrelevant to building social sensors. -SPS
%
%Whereas the above works addressed exploiting the detection of crisis
%events, the following works focused on descriptive studies on
%disaster. The studies discuss the behavior of Twitter users during a
%crisis \cite{vieweg,cheong,starbird} and do not address exploiting
%detection of crisis events. The studies investigated the use of social
%media during a crisis in order to identify information propagation
%properties, the social behavior of users (their retweeting behavior),
%information contributing to situational awareness, and the active
%players in communicating information. The behavioral information
%gleaned from these studies is exploited in this work to aid in the
%development of social sensors for detection of topics.
%
In another example use case to detect health epidemics such as
influenza, researchers build purpose-specific classifiers targeted to
this specific epidemic~\cite{culotta,aramaki}, e.g, by exploiting
knowledge of users' proximity and friendship along with the contageous
nature of influenza~\cite{sadilek}.  While these targeted event
detectors have the potential of providing high precision event
detection, they are highly specific to the target event and do not
easily generalize to learn arbitrary event-based or topic-based social
sensors as provided in this work.

\vspace{2mm}
\noindent {\bf Friend Sensors} 
are a fourth and final class of social sensors intended for early
event detection~\cite{sandy,garcia} by leveraging the concept of the
"friendship paradox"\cite{feld}, %~\footnote{On average, most people have
%fewer friends than their friends have.}),
to build user-centric social sensors.  We note that our topical social
sensors represent a \emph{superset} of friend sensors since our work
includes author features that the predictor may learn to use if this
proves effective for prediction.  However, as shown in our feature
analysis, user-based features are among the least informative feature
types for our topical social sensors suggesting that general social
sensors benefit from a wide variety of features well beyond those of
author features alone.

%In this regard, \citeauthor{garcia} provided a method for
%choosing sensor groups from friends of random sets of users to find
%more central individuals in order to enforce early detection. The
%central assumption made in this work is that a sensor group represents
%more central individuals, and individuals at the center of a network
%are more likely to become infected than randomly-chosen members of the
%population. As a result, \cite{garcia} argued that this selection
%process of sensor groups helps in the early detection of outbreaks.

%%%%%%% SOCIAL SENSOR PROJECT
%% NOTE: the citations below have been integrated into the text above... separate back out?
%
%Social sensor project \footcite{http://www.socialsensor.eu/} 
%\cite{aiello} compared six trending topic detection methods on three Twitter datasets differing in time scale and topic churn rate. The authors conclude that natural language processing techniques perform well on focused topics. However, techniques mining temporal distribution of concepts are needed to handle more heterogeneous streams.
%\cite{diplaris} defines social sensor as analyzing the dynamic and massive amount of information provided by user with the purpose of extracting unbiased trending topics and events in addition to using social connections for recommendation. The authors introduce concept of Dynamic Social Containers in this work to take advantage of aggregation of mining both the structure, content, and multimedia data to index and provide personalized, context-aware search.
%With the purpose of photo recommendation on social media websites, \cite{chiarandini} analyzed the user logs of pageviews, navigation patterns between photostreams. The authors used collaborative filtering method and built a stream transition graph to analyze common stream topic transitions to this end.

%%%%%%%%%%%%%%% TWITTER CURRENT SEARCH METHOD %%%%%%%%%%%%%%%%%
%% UNUSED
%\iffalse
%
%Twitter model: reverse indexes was built in MySQL, leveraging its concurrent transactions and B-tree data structures to support indexing and searching partitioned across multiple databases. Earlybird, a real-time reverse index based on Lucene, gave much better performance and memory efficiency than MySQL for real-time search. 
%"There is a lot of information on Twitter — on average, more than 2,200 new Tweets every second! During large events, for example the \#tsunami in Japan, this rate can increase by 3 to 4x. Often, users are interested in only the most memorable Tweets or those that other users engage with. In our new search experience, we show search results that are most relevant to a particular user. So search results are personalized, and we filter out the Tweets that do not resonate with other users."
%
%Supporting personalized search, they needed three types of signal: Static signals at indexing time, Resonance signals updated over time, Information about the searcher at search time. At indexing time, tweets are annotated with static information about the user and the language of the tweet's text. Dynamic updates, such as users' interactions with tweets are made over time. At query time, user's social graph is passed along the user's query. A specialized ranking function is used to combine relevance signals and the social graph for computing personalized relevance score for each tweet. The results consists of highest-ranking, most-recent tweets. The ranking function accesses the social graph and uses knowledge about the relationship between the searcher and the author of a tweet during ranking.
%\fi
%%%%%%%%%%%%%%%%%%%%%%%%%%%%%%%%%%%%%%%%%%%%%%%%%%%%%%
