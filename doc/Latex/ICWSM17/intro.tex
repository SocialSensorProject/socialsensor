%!TEX root = document.tex

\label{sec:introduction}

Social media sites such as Twitter present a double-edged sword for
users.  On one hand these sources contain a vast amount of novel and
topical content that challenge traditional news media sources in terms
of their timeliness and diversity.  Yet on the other hand they also
contain a vast amount of spam and otherwise low-value content for most
users' information needs where filtering out irrelevant content is
extremely time-consuming.  Hence, while it is widely acknowledged that
social media sources can be used as topical content sensors (indeed,
an entire European Union project was focused on related ``Social Sensor''
research\footnote{\texttt{http://www.socialsensor.eu/}}.),
automatically learning high-precision sensors (i.e., ranking and
retrieval methods) for arbitrary topics that generalize to future
unseen content remains an open question in the literature and
comprises the key problem we seek to address in this paper.

In this work, we contribute a novel supervised method for training
social sensors with minimal user curation by using a small seed set of
hashtags as topical proxies for automatic supervised data labeling.
Then we proceed to train supervised classification and ranking methods
to learn topical content from a large feature space of source users
and their locations, terms, hashtags, and mentions.  On a corpus of
over 800 million English Tweets collected from the Twitter streaming
API during 2013 and 2014 and covering 10 diverse topics ranging from
social issues to celebrity deaths to the ``Iran nuclear deal'', we
empirically show that two simple and efficiently trainable methods ---
logistic regression and naive Bayes --- generalize well to unseen
future topical content (including content with no hashtags) in terms
of their mean average precision (MAP) and Precision@$n$ for a range of
$n$.  Furthermore, we show that terms and locations are among the most
useful features --- surprisingly more so than hashtags, even though
hashtags were used to label the data.  And perhaps even more
surprisingly, the number of unique hashtags and tweets by a user
correlates more with their informativeness than their follower or
friend count.

%Overall, our feature analysis indicates that the most
%useful features are sometimes counter-intuitive and that in general
%learning methods may be much more effective than manual engineering
%for building topical social sensors.

In summary, this work fills a major gap in 
event detection and tracking from social media
%\eat{the literature of topical social sensors and } 
on identifying emerging topics from long-running themes with
%\eat{how to effectively and efficiently learn them given}
minimal user supervision.  Our results suggest that these
sensors generalize well to unseen future topical content and provide a
novel paradigm for the extraction of high-value content from social
media.

%Twitter hosts lots of information, on average more than $2,200$ new tweets every second. This can get up to $3$ to $4$ times increase during large events such as tsunami. \footnote{\hyperref[]{https://blog.twitter.com/2011/the-engineering-behind-twitter-s-new-search-experience}}
%\begin{itemize}
%\item Twitter is a vast sensor of content generated by latent phenonema (e.g., flu, political sentiment, elections, environment).
%\item Learning topical social sensors (politicians in NY, road conditions in Toronto) -- very broad topics for which its hard to manually specify a useful query.
%\item But there is interesting topical content and wouldn't it be cool if we could learn a social sensor for a targeted topic?
%\item Key insight is that hashtags are topical and can be used to bootstrap a supervised learning system that as we will show generalizes well beyond the seed hashtags.
%\item Conclusion is a new way to build topical real-time feeds that are otherwise difficult to do with existing Twitter tools (???).
%\end{itemize}
%section{Learning Topical Social Sensors}

%Start off with the questions that we want to answer in this section:
%
%- How to evaluate, labeling (problem of no supervised labels for tweets, indirect via hashtags as topical surrogates, leads to question of hashtag curation)?
%
%- Which classification algorithm is best / most robust for learning topical social sensors?
