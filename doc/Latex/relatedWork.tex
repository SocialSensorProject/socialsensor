%%%%%%%%%%%%%%% TWITTER CURRENT SEARCH METHOD %%%%%%%%%%%%%%%%%
\iffalse
Twitter Search : https://blog.twitter.com/2011/the-engineering-behind-twitter-s-new-search-experience

Twitter model: reverse indexes was built in MySQL, leveraging its concurrent transactions and B-tree data structures to support indexing and searching partitioned across multiple databases. Earlybird, a real-time reverse index based on Lucene, gave much better performance and memory efficiency than MySQL for real-time search. 
"There is a lot of information on Twitter — on average, more than 2,200 new Tweets every second! During large events, for example the \#tsunami in Japan, this rate can increase by 3 to 4x. Often, users are interested in only the most memorable Tweets or those that other users engage with. In our new search experience, we show search results that are most relevant to a particular user. So search results are personalized, and we filter out the Tweets that do not resonate with other users."

Supporting personalized search, they needed three types of signal: Static signals at indexing time, Resonance signals updated over time, Information about the searcher at search time. At indexing time, tweets are annotated with static information about the user and the language of the tweet's text. Dynamic updates, such as users' interactions with tweets are made over time. At query time, user's social graph is passed along the user's query. A specialized ranking function is used to combine relevance signals and the social graph for computing personalized relevance score for each tweet. The results consists of highest-ranking, most-recent tweets. The ranking function accesses the social graph and uses knowledge about the relationship between the searcher and the author of a tweet during ranking.
\fi
%%%%%%%%%%%%%%%%%%%%%%%%%%%%%%%%%%%%%%%%%%%%%%%%%%%%%%

This section documents existing research on the use of social media as a sensor for topic detection on social media. Herein, we focus on related research on both events and topics detection within social media. With the consideration that events are special type of topics and can be classified as such. To see how different works address topic detection on social media, we focus on three extensively researched types of topic detection: trending topic detection, specific event detection, and tweet recommendation. 

%%%%% TRENDING TOPIC DETECTION
The first overarching group of works reviewed herein focus on trending topic detection methods. The majority of works detecting trending topics use bursts as the indicator of events, where a burst is defined as a sudden change in posting rates of some keywords, hashtags, etc. These can further be divided into multiple categories based on how they use bursts to extract the event. The first category, clustering-based methods, focuses on the hypothesis that trends are topical and topics are defined by the collection of relevant content; hence trends can be detected by clustered content \cite{petrovic,ishikawa,murata,becker,tweetmotif,wangLee}. With more focus on machine learning methods, \cite{wei} proposed a graphical model to discover latent events clustered in the spatial, temporal and lexical dimensions, while \cite{yamamoto} focused on the task of multi-label classification of tweets into living aspects such as eating.%They use hierarchical estimation framework to estimate aspects of unknown tweets. This task is formulated into two phase of extracting topics from set of tweets using LDA and  calculating relevance between topics and aspects of tweets with computing Shannon entropy of each association.
The second category, term-based methods focuses on the hypothesis that topics can be detected by focusing on temporal patterns of terms/keywords independent of the content of documents \cite{mathioudakis,cuiZhang,zhaoSports,nichols}. The third category, query-based methods, focuses on the hypothesis that trending topics can be detected by measuring user defined criteria \cite{albakour,sakakiDrive}. The fourth category, network Structure-based method, focuses on the hypothesis that trending topics can be detected by studying the network structure of users \cite{budak}. The final category, hybrid method of \cite{diplaris} introduced concept of Dynamic Social Containers in this work to take advantage of aggregation of mining both the structure, content, and multimedia data to index and provide personalized, context-aware search. In this work, the authors defined social sensor as analyzing the dynamic and massive amount of information provided by user with the purpose of extracting unbiased trending topics and events in addition to using social connections for recommendation.

With the purpose of comparison of methods, \cite{aiello} evaluated six trending topic detection methods on three Twitter datasets differing in time scale and topic churn rate. The authors conclude that natural language processing techniques perform well on focused topics. However, techniques mining temporal distribution of concepts are needed to handle more heterogeneous streams.

However, trending topics detection methods are not targeted. Our method differs from trending topic detection methods in that we are focusing on a set of topics that cannot necessarily be detected using bursts.Thus, trending topics detection methods are of limited relevance to the work presented hereinafter.

%%%%% TARGETED SPECIFIC TOPIC DETECTION
The second overarching group of works focuses on detection of a specific targeted topic, such as a disaster or epidemic. In a predictive study by \cite{sandy}, the authors studied the network of users and focused on choosing the best groups of users in order to achieve lead-times i.e. faster detection of disastrous event (following the concept of "friendship paradox"\cite{feld} \footnote{On average, most people have fewer friends than their friends have}). \cite{sakakiEq2} used SVM classifier to detect earthquakes and employed a location estimation method such as Kalman Filtering for localizing it. The authors detected the occurrence of earthquakes through extracted statistical features e.g., the number and position of words in a tweet, keyword features and word context features from tweets.

Whereas the above works addressed exploiting the detection of crisis events, the following works focused on descriptive studies on disaster. The studies discuss the behavior of Twitter users during a crisis \cite{vieweg,cheong,starbird} and do not address exploiting detection of crisis events. The studies investigated the use of social media during a crisis in order to identify information propagation properties, the social behavior of users (their retweeting behavior), information contributing to situational awareness, and the active players in communicating information. The behavioral information gleaned from these studies is exploited in this work to aid in the development of social sensors for detection of topics.

To detect health epidemics, researchers used content-based and/or structure-based methods. The content-based methods of \cite{culotta} and \cite{aramaki} identified influenza-related tweets and correlated these tweets to United States Center for Disease Control (CDC) statistics on influenza, such as the infection and incubation rate. As for methodology, both works extracted bag-of-words as features, while the former employed single and multiple linear regression showing that multiple linear regression works better, while the latter employed SVM. Results indicated a high correlation between their estimation of influenza cases in early stages of an epidemic, and statistics from the CDC and Japan's Infection Disease Surveillance Center. The other approach to early detection of contagious outbreaks is to use structure-based methods, \cite{garcia} designed a sensor based on the friendship paradox concept for early detection of contagious outbreaks. In this regard, \citeauthor{garcia} provided a method for choosing sensor groups from friends of random sets of users to find more central individuals in order to enforce early detection. The central assumption made in this work is that a sensor group represents more central individuals, and individuals at the center of a network are more likely to become infected than randomly-chosen members of the population. As a result, \cite{garcia} argued that this selection process of sensor groups helps in the early detection of outbreaks.

On the other hand, hybrid method of \cite{sadilek}, exploited tweet content and the structural information of a user's network. The authors employed a semi supervised approach to learn a SVM classifier, using n-grams as features in order to detect ill individuals. Using co-location and friendship, the authors estimated the probability of physical interaction between healthy and sick people. This enabled them to study the effect of these two factors of social activity (co-location for contact network and friendship for social ties) on public health.

The limitations of these studies centers on the fact that the proposed methods are only valid for detecting a single topic. These methods used a primitive methods for curating the data e.g., querying keyword “earthquake”. In addition, there is no discussion within these works on how these methods can be generalized for other topics.

%%%%% TWEET RECOMMENDATION & RETWEET
Another set of studies have moved towards creating more generalizable methods. Using a dataset of 55,000 news articles and 121,000 tweets, \cite{Krestel} compared four different methods of language model, topic model, logistic regression, and boosting, to evaluate recommended tweets for a given news article.. \cite{Yan,chen} also focused on tweet recommendation. Their methods considered the user’s twitter profile, including tweet and retweet history, and social relations as features. Coupled with tweet popularity, the methods are able to generate tweet recommendations. With the purpose of photo recommendation on social media websites, \cite{chiarandini} analyzed the user logs of pageviews, navigation patterns between photostreams. The authors used collaborative filtering method and built a stream transition graph to analyze common stream topic transitions to this end.

On retweet prediction, \cite{can,xu,petrovicOsborne} used classification-based approaches using tweet-based and author-based features. However, \cite{can} took advantage of visual cues from images linked in the tweets, and \cite{xu} employed social-based features in addition to tweet author-based features. Different from the other two works, \cite{xu} performed the analysis from the perspective of individual users. \cite{petrovicOsborne} worked on retweet prediction of real-time tweeting with online learning algorithms and claimed that performance is dominated by social features, but that tweet features add a substantial boost. These studies showed that temporal features have a stronger effect on messages with low and medium volume of retweets compared to highly popular messages, and user activity features can further improve the performance marginally.

%%%%%%% SOCIAL SENSOR PROJECT
%Social sensor project \footcite{http://www.socialsensor.eu/} 
%\cite{aiello} compared six trending topic detection methods on three Twitter datasets differing in time scale and topic churn rate. The authors conclude that natural language processing techniques perform well on focused topics. However, techniques mining temporal distribution of concepts are needed to handle more heterogeneous streams.
%\cite{diplaris} defines social sensor as analyzing the dynamic and massive amount of information provided by user with the purpose of extracting unbiased trending topics and events in addition to using social connections for recommendation. The authors introduce concept of Dynamic Social Containers in this work to take advantage of aggregation of mining both the structure, content, and multimedia data to index and provide personalized, context-aware search.
%With the purpose of photo recommendation on social media websites, \cite{chiarandini} analyzed the user logs of pageviews, navigation patterns between photostreams. The authors used collaborative filtering method and built a stream transition graph to analyze common stream topic transitions to this end. 