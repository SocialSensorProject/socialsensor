%!TEX root = icwsm2016.tex

Social media sources such as Twitter represent a massively distributed
social sensor over a kaleidoscope of topics ranging from social and
political events to entertainment and sports news.
%However, given the
%continual evolution of social media content, querying for content from
%individual users or containing certain keywords or hashtags is often
%insufficient to retrieve the vast range of topical content available.
However, due to the overwhelming volume of content, it can be
difficult to identify novel and significant content within a broad
theme in a timely fashion.
% Not sure if we should mention hashtags -- doesn't this make retrieval trivial?
% But in general since this differs from our 10 themes below, maybe omit it?  -SPS
%
%such as \#obamacare or a recent \#twochild policy in China.  
%
%such as content involving the US
%healthcare system or the recent relaxation of the one-child policy in
%China.
To this end, this paper proposes a scalable and practical method to
automatically construct social sensors for generic topics.
Specifically, given minimal supervised training content from a user, we learn 
%address the task of automatically learning topical
%social sensors that generalize to future unseen content
%As a simple
%but critical insight, we leverage hashtags as proxies for topical
%content to automatically label a vast corpus of social media with only
%a small number of topical hashtags that users can easily curate.  With
%this labeled data in hand, we train a supervised learner to identify
to identify topical tweets 
from millions of features capturing content, user and social interactions on Twitter.
%over a large feature space given a small set of seed hashtags as proxies for
%topic labels.
On a corpus of over 800 million English Tweets
% or 40TB of Twitter data?  does not seem like 1 billions tweets though!
collected from the Twitter
streaming API during 2013 and 2014 and learning for 10 diverse themes 
ranging from social issues to celebrity deaths to the ``Iran nuclear
deal'', we empirically show that our learned social sensor
automatically generalizes to unseen future content 
%\eat{(including content with no hashtags)}
with high ranking and precision scores.  Furthermore, we provide an
extensive analysis of features and feature types across different
topics that reveals, for example, that (1)~largely independent of
topic, simple terms are the most informative feature followed by
location features and that (2)~the number of unique hashtags and
tweets by a user correlates more with their informativeness than their
follower or friend count.  In summary, this work provides a novel,
effective, and efficient way to learn topical social sensors requiring
minimal user curation effort and offering strong generalization performance
for identifying future topical content.

%Twitter represents a massively distributed social sensor of a rich
%underlying topic space that drives its content generation.  Yet
%Twitter content is so diverse, decentralized, and dynamic in nature,
%that it is hard to automatically aggregate this topical content.  To
%address this need, we provide a novel way of learning topical social
%sensors on Twitter that learn from a provided set of topical hashtags
%and generalize to identify topical tweets with previously unseen tags.
%These learning social sensors leverage a variety of user-based,
%hashtag-based, term-based, and location-based features for
%distinguishing topical from non-topical tweets; we further analyze
%these features to understand which features are most useful and why.
%We further assess general global topical trends and how our learning
%sensors are able to follow these trends by drawing from a rich variety
%of sources on the Twittersphere to enable a first generation of
%learning social sensors for Twitter.
